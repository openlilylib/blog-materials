\documentclass{ulPlainA4}
\usepackage{fontspec}
\usepackage{lilyglyphs}
\usepackage{hologo}
\usepackage{microtype}
\setmonofont[%
	Scale=MatchLowercase]{Inconsolata}

\begin{document}
Von noch größerer Bedeutung für den Entstehungsprozess dieser Edition war jedoch LilyPonds textbasiertes Eingabekonzept.
Durch dieses und die Verwendung des -- ebenfalls textbasierten Textsatzsystems \hologo{LuaLaTeX} -- war es uns möglich, die gesamte Vorbereitung der Edition wie ein Softwareentwicklungsprojekt unter \emph{Versionskontrolle} durchzuführen.
Dies erlaubte es, Arbeitsabläufe und insbesondere die Interaktion zwischen Herausgebern und Notensetzer in einer mir zuvor unvorstellbaren Weise effizient und zuverlässig zu strukturieren.
Ein erfreuliches »Nebenprodukt« ist \lilyglyphs{}%
\footnote{\texttt{http://www.ctan.org/pkg/lilyglyphs}}.
Mit diesem von mir für den Revisionsbericht entwickelten -- und inzwischen in den gängigen \LaTeX-Distributionen enthaltenen -- Paket können beliebige Notationselemente im Fließtext von Textdokumenten verwendet werden.

Dank gilt den Mitgliedern der \texttt{lilypond-user} Mailingliste, die immer wieder unschätzbare Anregungen zu technischen Lösungen gegeben, aber auch allgemeine notensetzerische oder orthotypografische Fragen erörtert haben.
\end{document}